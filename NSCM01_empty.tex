%  Revised on Oct 24, 1999 (done on Aug 20)--spring.
%

%\documentstyle[12pt]{article}
\documentclass[12pt,a4paper,openany]{article}

%\usepackage[square, comma, sort&compress]{natbib}
\usepackage{cite}
%\pagestyle{foot}
\textheight=255mm
%\textheight=240mm
%\textwidth=160mm
\textwidth=185mm
\topmargin=-2.5cm
%\topmargin=0cm
\oddsidemargin=-1.5cm
%\parskip 3mm
%\setcounter{page}{1}

\usepackage{cite}
\usepackage{amsmath}
\usepackage{amsfonts}
\usepackage{amssymb}
\usepackage{fancyhdr}
\usepackage{multirow}
\pagestyle{fancy}
\fancyhead{} % 清除所有頁首設定
\fancyfoot{}
\renewcommand{\headrulewidth}{0pt}
\renewcommand{\footrulewidth}{0pt}
\setlength{\headwidth}{\textwidth}

\rfoot{共~~\pageref{LastPage}~~頁~~第{~~\thepage~~}頁}
\lfoot{研究成果}
%\rfoot{ 共 \pageref{LastPage} 頁 第  \thepage   頁} 

\usepackage{fontspec}
%\usepackage[BoldFont, ItalicFont]{xeCJK}
\usepackage{xeCJK}
%\usepackage[ItalicFont]{xeCJK}
\setCJKmainfont{標楷體}
%\setCJKitalicfont{華康行書體}
\XeTeXlinebreaklocale "zh"             %這兩行一定要加,中文才能自動換行
\XeTeXlinebreakskip = 0pt plus 1pt     %這兩行一定要加,中文才能自動換行



% define a macro
%\def\texpsfig#1#2#3{\vbox{\kern #3\hbox{\special{psfile=#1}\kern #2}}\typeout{(#1)}}

\begin{document}
\newtheorem{thm}{Theorem}
\newtheorem{lem}{Lemma}
\newtheorem{cor}{Corollary}
\newtheorem{prop}{Proposition}
\newtheorem{conj}{Conjecture}
%\input epsf   % load the epsf library

\baselineskip 7mm

\begin{center}
{\bf \large  {\color {red}106} 年度自然司專題計畫主持人近五年研究成果 }
\end{center}


姓名:     職稱:      服務機關系所:        

\begin{enumerate}
 \item [一、]近五年內(2012/1/1~2016/12/31)已出版之最具代表性研究成果至多六篇,擇其中五篇電子檔上傳。
 {\small (請依序填寫:姓名,著作名稱,發表年份,期刊,卷數,頁次,IF,▲:被引用次數,並以*號註記該篇所有的通訊作者)}\newline
 {\color{red}\small (請務必更新個人資料表C302-C303,未來審查時將以該表之內容為準)}
  % 輸入 第一項內容 
  \begin{enumerate}
  \item[1.]  Lin
  \item[2.]  Lin
  \end{enumerate}
 
 
 \item [二、]
近五年內獲獎情形及重要會議邀請演講至多五項。

% 輸入 第二項內容 

 \item [三、] 近五年內其他資料:擔任國際重要學術學會理監事、國際知名學術期刊編輯或評審委員等。
 
 \item [四、] 請簡述上述代表性研究成果之個人重要貢獻(至多一頁)。
 
 \item [五、] 研發成果智慧財產權及其應用績效:
 \begin{itemize}
 \item[1.]  專利:請填入目前仍有效之專利。「類別」請填入代碼:
  (A)發明專利(B)新型專利(C)新式樣專利。
  \begin{table}[h!]
  \centering
  \begin{tabular}{|l|l|l|l|l|l|l|l|}
  \hline
  \multirow{2}{*}{類別} & \multirow{2}{*}{專利名稱} & \multirow{2}{*}{國別} & \multirow{2}{*}{專利號碼} & \multirow{2}{*}{發明人} & \multirow{2}{*}{專利權人} & 專利核准 & \multirow{2}{*}{科技部計畫編號} \\ 
   &   &  &  &  &  & 日    期 &    \\ \hline
   &  &  &  &  &  &   &  \\ \hline
   &  &  &  &  &  &   &  \\ \hline
   &  &  &  &  &  &   &  \\ \hline
   &  &  &  &  &  &   &  \\ \hline
  \end{tabular}
  \end{table}
 
  \item[2.] 技術移轉:
   \begin{table}[h!]
   \centering
   \begin{tabular}{|l|l|l|l|l|l|l|}
   \hline
   \multirow{2}{*}{技術名稱} & \multirow{2}{*}{專利名稱} & \multirow{2}{*}{授權單位} & \multirow{2}{*}{被授權單位} & \multirow{2}{*}{簽約日期}  & 權利金,衍生 & \multirow{2}{*}{科技部計畫編號} \\ 
    &   &  &  &  &   利益金等 &    \\ \hline
    &  &  &  &  &  &     \\ \hline
    &  &  &  &  &  &     \\ \hline
 
   \end{tabular}
   \end{table}
 \end{itemize}
 
 
\end{enumerate}
 


% 輸入 第三項內容 

 

\label{LastPage}
\end{document}
