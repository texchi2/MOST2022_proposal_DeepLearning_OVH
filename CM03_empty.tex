%  Revised on Oct 24, 1999 (done on Aug 20)--spring.
%

%\documentstyle[12pt]{article}
%\documentclass[12pt,a4paper,openany]{article}
\documentclass[12pt,a4paper]{article}

%\usepackage[square, comma, sort&compress]{natbib}
%\pagestyle{foot}
\textheight=255mm
%\textheight=240mm
%\textwidth=160mm
\textwidth=175mm
\topmargin=-2.5cm
%\topmargin=0cm
\oddsidemargin=-1cm
%\parskip 3mm
%\setcounter{page}{1}

\usepackage{epsfig}
\usepackage{cite}
\usepackage{amsmath}
\usepackage{amsfonts}
\usepackage{amssymb}
\usepackage{fancyhdr}
\pagestyle{fancy}
\fancyhead{} % 清除所有頁首設定
\fancyfoot{}
\renewcommand{\headrulewidth}{0pt}
\renewcommand{\footrulewidth}{0pt}
\setlength{\headwidth}{\textwidth}

\lfoot{表 CM03}
\rfoot{共~~\pageref{LastPage}~~頁~~第{~~\thepage~~}頁}


\usepackage{fontspec}
\usepackage[BoldFont]{xeCJK}
\setCJKmainfont{標楷體}
\XeTeXlinebreaklocale "zh"             %這兩行一定要加,中文才能自動換行
\XeTeXlinebreakskip = 0pt plus 1pt     %這兩行一定要加,中文才能自動換行

%\setmainfont{新細明體}                    %設定全文使用的字型
%\newfontfamily{\A}{Times New Roman}       %設定要選用的第一種字型
%\newfontfamily{\B}{標楷體}                  %設定要選用的第二種字型
%\newfontfamily{\C}{新細明體}


% define a macro
%\def\texpsfig#1#2#3{\vbox{\kern #3\hbox{\special{psfile=#1}\kern #2}}\typeout{(#1)}}

\begin{document}
\newtheorem{thm}{Theorem}
\newtheorem{lem}{Lemma}
\newtheorem{cor}{Corollary}
\newtheorem{prop}{Proposition}
\newtheorem{conj}{Conjecture}
\input epsf   % load the epsf library

\baselineskip 7mm

\noindent
\textbf{\large  三、研究計畫內容(以中文或英文撰寫):}
\begin{description}
\item[(一)]  研究計畫之背景。請詳述本研究計畫所要探討或解決的問題、重要性、預期影響性及國內外有關本計畫之研究情況、重要參考文獻之評述等。如為連續性計畫應說明上年度研究進度。


\item[(二)] 研究方法、進行步驟及執行進度。請分年列述:1.本計畫採用之研究方法與原因。2.預計可能遭遇之困難及解決途徑。3.重要儀器之配合使用情形。4.如為須赴國外或大陸地區研究,請詳述其必要性以及預期效益等。

\item[(三)] 預期完成之工作項目及成果。請分年列述:1.預期完成之工作項目。2.對於參與之工作人員,預期可獲之訓練。3.預期完成之研究成果(如期刊論文、研討會論文、專書、技術報告、專利或技術移轉等質與量之預期成果)。4.學術研究、國家發展及其他應用方面預期之貢獻。

\item[(四)] 整合型研究計畫說明。如為整合型研究計畫請就以上各點分別說明與其他子計畫之相關性。


\end{description}


\label{LastPage}
\end{document}
